\chapter{Introduction}
This thesis explores how global IPv6 addressing increases the attack surface of the Internet of Things (IoT) environment. The study combines not only a comprehensive literature review but also an empirical assessment of default system configurations through controlled laboratory demonstrations, highlighting how the adoption of new network standards can potentially affect the security of millions of end devices.
%Should I focus in domestic networks?

\section{The IPv4 to IPv6 Transition}is the result of an inherent limitation in the design of IPv4’s address space, which only supports $2^{32}$ unique addresses. A boundary already exceeded by the estimated 16 billion IoT devices alone in 2024~\cite{iotanalytics2024}. IPv6 mitigates this limitation by implementing a 128-bit architecture, offering a virtually infinite address space. Despite its technical advantage, the global adoption has been gradual due to the current scale of IPv4-based infrastructures and the cost of migration. Therefore, technologies like Network Address Translation (NAT) were adopted to extend the utility of IPv4.


\section{Loss of NAT and IoT Exposure}


\section{Research Objectives \& Questions}


\section{Contributions}


\section{Thesis Structure}

